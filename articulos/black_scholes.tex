\documentclass[12pt]{article}
	\usepackage[total={18cm,27cm},top=2cm, left=2cm, bottom=2cm, right=2cm]{geometry}

\title{Deducci\'{o}n de las f\'{o}rmulas de Black-Scholes mediante valor esperado del pago futuro \footnote{\LaTeX}}
\author{Alexis S\'{a}nchez Tello de Meneses}
\date{4 Septiembre 2014}

\begin{document}

\maketitle

\section{Abstract}

Se desarrollar\'{a} a partir del modelo de evoluci\'{o}n {\em log-normal} para un subyacente, las f\'{o}rmulas de
Black-Scholes para el precio de opciones plain vanilla (call/put) europeas, as\'{i} mismo, mediante derivaci\'{o}n directa
de las f\'{o}rmulas con respecto a sus par\'{a}metros obtendremos las griegas m\'{a}s representativas.

\section{Modelo {\em log-normal} del subyacente.}

Se asume que la evoluci\'{o}n del precio del subyacente (precio de una de acci\'{o}n en el mercado de
renta variable), $S$, es un proceso estoc\'{a}stico continuo y {\em log-normal}. La descripci\'{o}n
matem\'{a}tica de este proceso queda recogida en la siguiente ecuaci\'{o}n diferencial estoc\'{a}stica.
\newline

\begin{eqnarray}
	dS_{t} = \mu S_{t}dt + \sigma S_{t}dW_{t}
\end{eqnarray}
\newline

Aqu\'{i}, $dS_{t}$ es $S_{t+dt}-S_{t}$. La deriva del proceso ser\'{i}a $\mu$, que coincidir\'{a} con el tipo de 
inter\'{e}s continuo y anual libre de riesgo de la cuenta bancaria, escogi\'{e}ndose esta como numerario, para que el
proceso del logaritmo del subyacente sea una martingala. La volatilidad anualizada del subyacente ser\'{a} $\sigma$,
siendo $dW_{t}$ un salto gaussiano de media cero y desviaci\'{o}n t\'{i}ìca $\sqrt{dt}$.
\newline

Si incorporamos a (1) los pagos de dividendo, de manera continua, con una tasa anual $\delta$, el subyacente, al pasar del
valor $S_{t}$ en $t$ al valor $S_{t+dt}$ en $t+dt$, disminuye su valor en la cuant\'{i}a $\delta S_{t}$, que es 
justamente el dividendo que se acaba de repartir. La ecuaci\'{o}n (1) corregida con el pago de dividendos, quedar\'{i}a de
la forma:
\newline

\begin{eqnarray}
	dS_{t} = \mu S_{t}dt - \underbrace{\delta S_{t}dt}_{dividendo} + \sigma S_{t}dW_{t}
\end{eqnarray}

\subsection{Integraci\'{o}n por el lema de \^{I}to}

Para obtener la evoluci\'{o}n del subyacente en funci\'{o}n del tiempo y de la variable aletoria normal est\'{a}ndar,
efectuaremos el cambio de variable $S^{'}_{t}=\ln S_{t}$ en la ecuaci\'{o}n (2), y aplicaremos el lema de \^{I}to para 
calcular la diferencial de la nueva variable.
\newline

El lema de \^{I}to para una funci\'{o}n que s\'{o}lo depende del subyacente {em (i.e.\ $S_{t}^{'}=f(S_{t})$)} tendr\'{i}a
la siguiente forma:
\newline

\begin{eqnarray}
	dS_{t}^{'} & = & \frac{df}{dS_{t}}dS_{t} + \frac{1}{2}\frac{d^{2}f}{dS_{t}^{2}}dS_{t}^{2} + \ldots
\end{eqnarray} 

Usando $S_{t}^{'} = f(S_{t}) = \ln(S_{t})$ en el desarrollo en serie, y qued\'{a}ndonos hasta t\'{e}rminos de segundo orden en $dS_{t}$:
\newline

\begin{eqnarray}
	dS_{t}^{'}	& = & \frac{dS_{t}}{S_{t}} - \frac{1}{2}\cdot\frac{1}{S_{t}^{2}}dS_{t}^{2} + \ldots \nonumber \\
			& = & \left( \mu - \delta \right) dt + \sigma dW_{t} - \frac{1}{2}\sigma^{2}dW_{t}^{2} + \ldots \nonumber \\
			& = & \left( \mu - \delta - \frac{1}{2}\sigma^{2} \right)dt + \sigma dW_{t} 		 
\end{eqnarray}

Usando el anterior cambio de variable reducimos la ecuaci\'{o}n diferencial estoc\'{a}stica log-normal a una ecuaci\'{o}n, de saltos normales de
desviaci\'{o}n t\'{i}pica $\sigma \sqrt{dt}$. Esta ecuaci\'{o}n se puede {\em integrar} directamente dando como resultado la evoluci\'{o}n
temporal del logaritmo del subyacente. (Sabiendo que en el sentido de variables aleatorias, a efectos pr\'{a}cticos de valores esperados y momentos, se cumple que $\sigma dW_{t} \simeq \sigma \sqrt{dt} X \left( 0,1 \right)$ , siendo 
$ X \left( 0, 1 \right)$, la variable aleatoria normal est\'{a}ndar). Sustituy\'{e}ndolo en la ecuaci\'{o}n anterior, proporciona...).
\newline

\begin{equation}
	S_{T}^{'} = S_{t}^{'} + \left(\mu - \delta - \frac{\sigma^{2}}{2}\right) \tau + \sigma\sqrt{\tau}X(0,1)
\end{equation}

Aqu\'{i} $T$ representa el tiempo final de la evoluci\'{o}n de la funcion de cambio de variable del subyacente, y aqui $\tau$ es 
$T - t$.
\newline

Deshaciendo el cambio de variable en (5), obtenemos la evoluci\'{o}n del subyacete.
\newline

\begin{equation}
	S_{T}=S_{t}e^{\left( \mu - \delta - \frac{\sigma^{2}}{2} \right)\tau + \sigma\sqrt{\tau}X\left( 0,1\right)}
\end{equation}
\newpage

\section{Valoraci\'{o}n de una Call Europea}
El precio de una call europea lo deducimos como el valor esperado del pago a vencimiento, descontando desde su fecha hasta
la fecha de valoraci\'{o}n. Supongamos que el vencimiento es en $T$ y la fecha de valoraci\'{o}n es $t$, denotemos $\tau = T - t$ . 
El valor esperado del pago futuro sera

\begin{equation}
	E\left( e^{-r\tau} \left[ S_{T} - K \right] ^{+} \right) = \frac{1}{\sqrt{2\pi}} \int_{-\infty}^{+\infty}e^{-r\tau}
	\left[ S_{T} - K \right] ^{+} e^{-\frac{1}{2}x^{2}}dx
\end{equation}


\end{document}
