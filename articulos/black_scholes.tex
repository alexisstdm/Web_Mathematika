\documentclass[12pt]{article}
	\usepackage[total={18cm,27cm},top=2cm, left=2cm, bottom=2cm, right=2cm]{geometry}

\title{Deducci\'{o}n de las f\'{o}rmulas de Black-Scholes mediante valor esperado del pago futuro \footnote{\LaTeX}}
\author{Alexis S\'{a}nchez Tello de Meneses}
\date{4 Septiembre 2014}

\begin{document}

\maketitle

\section{Abstract}

Se desarrollar\'{a} a partir del modelo de evoluci\'{o}n {\em log-normal} para un subyacente, las f\'{o}rmulas de
Black-Scholes para el precio de opciones plain vanilla (call/put) europeas, as\'{i} mismo, mediante derivaci\'{o}n directa
de las f\'{o}rmulas con respecto a sus par\'{a}metros obtendremos las griegas m\'{a}s representativas.

\section{Modelo {\em log-normal} del subyacente.}

Se asume que la evoluci\'{o}n del precio del subyacente (precio de una de acci\'{o}n en el mercado de
renta variable), $S$, es un proceso estoc\'{a}stico continuo y {\em log-normal}. La descripci\'{o}n
matem\'{a}tica de este proceso queda recogida en la siguiente ecuaci\'{o}n diferencial estoc\'{a}stica.
\newline

\begin{eqnarray}
	dS_{t} = \mu S_{t}dt + \sigma S_{t}dW_{t}
\end{eqnarray}
\newline

Aqu\'{i}, $dS_{t}$ es $S_{t+dt}-S_{t}$. La deriva del proceso ser\'{i}a $\mu$, que coincidir\'{a} con el tipo de 
inter\'{e}s continuo y anual libre de riesgo de la cuenta bancaria, escogi\'{e}ndose esta como numerario, para que el
proceso del logaritmo del subyacente sea una martingala. La volatilidad anualizada del subyacente ser\'{a} $\sigma$,
siendo $dW_{t}$ un salto gaussiano de media cero y desviaci\'{o}n t\'{i}ìca $\sqrt{dt}$.
\newline

Si incorporamos a (1) los pagos de dividendo, de manera continua, con una tasa anual $\delta$, el subyacente, al pasar del
valor $S_{t}$ en $t$ al valor $S_{t+dt}$ en $t+dt$, disminuye su valor en la cuant\'{i}a $\delta S_{t}$, que es 
justamente el dividendo que se acaba de repartir. La ecuaci\'{o}n (1) corregida con el pago de dividendos, quedar\'{i}a de
la forma:
\newline

\begin{eqnarray}
	dS_{t} = \mu S_{t}dt - \underbrace{\delta S_{t}dt}_{dividendo} + \sigma S_{t}dW_{t}
\end{eqnarray}

\subsection{Integraci\'{o}n por el lema de \^{I}to}

Para obtener la evoluci\'{o}n del subyacente en funci\'{o}n del tiempo y de la variable aletoria normal est\'{a}ndar,
efectuaremos el cambio de variable $S^{'}_{t}=\ln S_{t}$ en la ecuaci\'{o}n (2), y aplicaremos el lema de \^{I}to para 
calcular la diferencial de la nueva variable.
\newline

El lema de \^{I}to para una funci\'{o}n que s\'{o}lo depende del subyacente {em (i.e.\ $S_{t}^{'}=f(S_{t})$)} tendr\'{i}a
la siguiente forma:
\newline

\begin{eqnarray}
	dS_{t}^{'} & = & \frac{df}{dS_{t}}dS_{t} + \frac{1}{2}\frac{d^{2}f}{dS_{t}^{2}}dS_{t}^{2} + \ldots
\end{eqnarray} 

Usando $S_{t}^{'} = f(S_{t}) = \ln(S_{t})$ en el desarrollo en serie, y qued\'{a}ndonos hasta t\'{e}rminos de segundo orden en $dS_{t}$:
\newline

\begin{eqnarray}
	dS_{t}^{'}	& = & \frac{dS_{t}}{S_{t}} - \frac{1}{2}\cdot\frac{1}{S_{t}^{2}}dS_{t}^{2} + \ldots \nonumber \\
			& = & \left( \mu - \delta \right) dt + \sigma dW_{t} - \frac{1}{2}\sigma^{2}dW_{t}^{2} + \ldots \nonumber \\
			& = & \left( \mu - \delta - \frac{1}{2}\sigma^{2} \right)dt + \sigma dW_{t} 		 
\end{eqnarray}

Usando el anterior cambio de variable reducimos la ecuaci\'{o}n diferencial estoc\'{a}stica log-normal a una ecuaci\'{o}n, de saltos normales de
desviaci\'{o}n t\'{i}pica $\sigma \sqrt{dt}$. Esta ecuaci\'{o}n se puede {\em integrar} directamente dando como resultado la evoluci\'{o}n
temporal del logaritmo del subyacente. (Sabiendo que en el sentido de variables aleatorias, a efectos pr\'{a}cticos de valores esperados y momentos, se cumple que $\sigma dW_{t} \simeq \sigma \sqrt{dt} X \left( 0,1 \right)$ , siendo 
$ X \left( 0, 1 \right)$, la variable aleatoria normal est\'{a}ndar). Sustituy\'{e}ndolo en la ecuaci\'{o}n anterior, proporciona...).
\newline

\begin{equation}
	S_{T}^{'} = S_{t}^{'} + \left(\mu - \delta - \frac{\sigma^{2}}{2}\right) \tau + \sigma\sqrt{\tau}X(0,1)
\end{equation}

Aqu\'{i} $T$ representa el tiempo final de la evoluci\'{o}n de la funcion de cambio de variable del subyacente, y aqui $\tau$ es 
$T - t$.
\newline

Deshaciendo el cambio de variable en (5), obtenemos la evoluci\'{o}n del subyacete.
\newline

\begin{equation}
	S_{T}=S_{t}e^{\left( \mu - \delta - \frac{\sigma^{2}}{2} \right)\tau + \sigma\sqrt{\tau}X\left( 0,1\right)}
\end{equation}
\newpage

\section{Valoraci\'{o}n de una Call Europea}
El precio de una call europea lo deducimos como el valor esperado del pago a vencimiento, descontando desde su fecha hasta
la fecha de valoraci\'{o}n. Supongamos que el vencimiento es en $T$ y la fecha de valoraci\'{o}n es $t$, denotemos $\tau = T - t$ . 
El valor esperado del pago futuro sera

\begin{equation}
	E\left( e^{-r\tau} \left[ S_{T} - K \right] ^{+} \right) = \frac{1}{\sqrt{2\pi}} \int_{-\infty}^{+\infty}e^{-r\tau}
	\left[ S_{T} - K \right] ^{+} e^{-\frac{1}{2}x^{2}}dx
\end{equation}
\newline

Aqu\'{i} $r$ es el inter\'{e}s libre de riesgo y $\left[S_{T}-K\right] ^{+}$ representa la funci\'{o}n de pago en tiempo $T$, que ser\'{a} $S_{T}-K$
si $S_{T}-K>0$ o $0$ en caso contrario.
\newline

Sustituyendo $S_{T}$ por la expresi\'{o}n de su evoluci\'{o}n en funci\'{o}n de $\tau = T - t$, y teniendo en cuenta que $x$ es $X\left(0,1
\right)$, esto es, la variable aleatoria normal est\'{a}ndar de media $0$ y desviaci\'{o}n t\'{i}pica $1$, obtenemos:
\newline

\begin{equation}
	E\left( e^{-r\tau} \left[ S_{T} - K \right] ^{+} \right) = \frac{1}{\sqrt{2\pi}}\int_{-\infty}^{+\infty}e^{-r\tau}
	\left[S_{t}e^{\left[\left(\mu-\delta-\frac{\sigma^{2}}{2}\right)t+\sigma\sqrt{t}x\right]}-K\right]^{+}e^{-\frac{1}{2}x^{2}}
	dx	
\end{equation}
\newline

Forzamos que el par\'{a}metro $\mu$ del modelo sea igual a $r$ para evitar oprtunidad de arbitraje. Por tanto la ecuaci\'{o}n anterior queda
de la forma:
\newline

\begin{equation}
	E\left( e^{-r\tau} \left[ S_{T} - K \right] ^{+} \right) = \frac{1}{\sqrt{2\pi}}\int_{-\infty}^{+\infty}e^{-\delta\tau}
	\left[S_{t}e^{\sigma\sqrt{\tau}x-\frac{\sigma^{2}}{2}\tau}-Ke^{-(r-\delta)\tau}\right]^{+}e^{-
	\frac{1}{2}x^{2}}dx
\end{equation}
\newline

El dominio de integraci\'{o}n de $(9)$ se puede descomponer en dos intervalos; uno en el que la funci\'{o}n valor positivo es distinta de
cero, y otro en el que es cero, con lo cual, como l\'{i}mite inferior de integraci\'{o}n sustituimos $-\infty$ por $x_{0}$, 
que es como denominamos al valor que divide la recta real en los dos intervalos. Vamos a calcular $x_{0}$ .
\newline

\begin{equation}
	S_{t}e^{\sigma\sqrt{\tau}x-\frac{\sigma^{2}}{2}\tau}-Ke^{-\left(r-\delta\right)\tau} \geq 0
\end{equation}
\newline

Veamos a partir de que valor de $x$ se cumple $(10)$. Tomando logaritmos y reordenando la inecuaci\'{o}n...
\newline

\begin{eqnarray}
	\ln S_{t} +x\sigma\sqrt{\tau} - \frac{\sigma^{2}}{2}\tau & \geq & \ln K - \left(r - \delta \right) \tau \nonumber \\
	x\sigma\sqrt{\tau} - \frac{\sigma^{2}}{2}\tau & \geq & -\ln S_{t} + \ln K - \left( r - \delta \right) \tau \nonumber \\
	x\sigma\sqrt{\tau} - \frac{\sigma^{2}}{2}\tau & \geq & -\left(\ln\left(\frac{S_{t}}{K}\right) + \ln\left(e^{\left(r-
	\delta\right)\tau}\right)\right) \nonumber \\
	x & \geq & - \frac{1}{\sigma\sqrt{\tau}} \ln\left(\frac{S_{t}}{Ke^{-\left(r-\delta\right)\tau}}\right) + 
	\frac{\sigma\sqrt{\tau}}{2}
\end{eqnarray}
\newline

El t\'{e}rmino derecho de $(10)$ es el valor frontera $x_{0}$ buscado. Por tanto la integral $(9)$ se reduce al intervalo de $x_{0}$ a
$-\infty$ .
\newline

\begin{equation}
	E\left( e^{-r\tau} \left[ S_{T} - K \right] ^{+} \right) = \frac{1}{\sqrt{2\pi}}\int_{-x_{0}}^{+\infty}e^{-\delta\tau}
	\left[S_{t}e^{\sigma\sqrt{\tau}x-\frac{\sigma^{2}}{2}\tau}-Ke^{-(r-\delta)\tau}\right]^{+}e^{-
	\frac{1}{2}x^{2}}dx
\end{equation}
\newline

Esto se descompone en la suma de dos integrales que pasamos a llamar $I_{1}$ e $I_{2}$ .
\newline

\begin{equation}
	\underbrace{\frac{1}{\sqrt{2\pi}}S_{t}e^{-\delta\tau}\int_{x_{0}}^{+\infty}e^{-\left(\frac{1}{2}x^{2}-\sigma\sqrt{\tau}x
	 + \frac{\sigma^{2}}{2}\tau\right)}dx}_{I_{1}} - \underbrace{\frac{1}{\sqrt{2\pi}}Ke^{-r\tau}\int_{x_0}^{+\infty}
	e^{-\frac{1}{2}x^{2}}dx}_{I_{2}}
\end{equation}
\newline

Resolvemos primero $I_{2}$ :

\begin{eqnarray}
	I_{2} 	& = & -\frac{1}{\sqrt{2\pi}}Ke^{-r\tau}\int_{x_{0}}^{+\infty}e^{-\frac{1}{2}x^{2}}dx \nonumber \\
	      	& = & -\frac{1}{\sqrt{2\pi}}Ke^{-r\tau}\int_{-\infty}^{-x_{0}}e^{-\frac{1}{2}x^{2}}dx
\end{eqnarray}

El \'{u}ltimo paso en el desarrollo de arriba es debido a que el integrando es una funci\'{o}n par. Denotemos $d_{-} = -x_{0}$. Entonces.
\newline

\begin{equation}
	d_{-} = \frac{1}{\sqrt{2\pi}}\ln\left(\frac{S_{t}}{Ke^{-\left(r-\delta\right)\tau}}\right) - 
	\frac{1}{2}\sigma\sqrt{\tau}	
\end{equation}
\newline

Por tanto tenemos que
\newline

\begin{equation}
	I_{2} = -\frac{1}{\sqrt{2\pi}}Ke^{-r\tau}\int_{-\infty}^{d_{-}}e^{-\frac{1}{2}x^{2}}dx
\end{equation}
\newline

Pero $-\frac{1}{\sqrt{2\pi}}\int_{-\infty}^{d_{-}}e^{-\frac{1}{2}x^{2}}dx$ es la funci\'{o}n distribuci\'{o}n normal est\'{a}ndar
de $d_{-}$. Finalmente queda:
\newline

\begin{equation}
	I_{2} = -Ke^{-r\tau}\Phi\left(d_{-}\right)
\end{equation}

S\'{o}lo queda calcular $I_{1}$ para obtener el precio de la opci\'{o}n. En $(13)$ se ve claramente que $x^{2}-2\sigma\sqrt{\tau}x+
\sigma^{2}\tau=\left(x-\sigma\sqrt{\tau}\right)^{2}$, quedando:
\newline

\begin{equation}
	I_{1} = \frac{1}{\sqrt{2\pi}}S_{t}e^{-\delta\tau}\int_{x_{0}}^{+\infty}e^{-\frac{1}{2}\left(x-\sigma\sqrt{\tau}\right)^{2}}dx
\end{equation}
\newline

En $(18)$ efectuamos el cambio de variable $x^{'}=\left(x-\sigma\sqrt{\tau}\right)$ con $dx^{'}=dx$, pasando a ser el l\'{i}mite 
inferior de la integral $x_{0}-\sigma\sqrt{\tau}$.
\newline

\begin{equation}
	I_{1}  	 =  S_{t}e^{-\delta\tau}\frac{1}{\sqrt{2\pi}}\int_{x_{0}-\sigma\sqrt{\tau}}^{+\infty}
			e^{-\frac{1}{2}x^{'^{2}}}dx^{'}
\end{equation}
\newline

Como $e^{-\frac{1}{2}x^{'^{2}}}$ es una funci\'{o}n par, podemos cambiar los signos y el orden de los l\'{i}mites de integraci\'{o}n. Haciendo
$d_{+}=-x_{0}+\sigma\sqrt{\tau}$, nos queda
\newline

\begin{eqnarray}
	I_{1} & = & S_{t}e^{-\delta\tau}\Phi\left(d_{+}\right) \nonumber \\
	I_{2} & = & -Ke^{-r\tau}\Phi\left(d_{-}\right) \nonumber \\
	d_{\pm} & = & \frac{1}{\sigma\sqrt{\tau}}\ln\left(\frac{S_{t}}{Ke^{-\left(r-\delta\right)\tau}}\right)\pm
			\frac{1}{2}\sigma\sqrt{\tau}
\end{eqnarray}
\newline

EL precio de la call europea en tiempo $t$ es:
\newline

\begin{equation}
	C\left(S_{t}\right) = S_{t}e^{-\delta\tau}\Phi\left(d_{+}\right) - Ke^{-r\tau}\Phi\left(d_{-}\right)
\end{equation}
\newline

Realizando el desarrollo anterior, pero utilizando el payoff de la put en tiempo $T$, $\left[K-S_{T}\right]^{+}$, obtenemos
la f\'{o}rmula del precio en tiempo $t\leq T$.
\newline
\begin{equation}
	P\left(S_{t}\right) = Ke^{-r\tau}\Phi\left(-d_{-}\right) - S_{t}e^{-\delta\tau}\Phi\left(-d_{+}\right)
\end{equation}
\newline

\section{C\'{a}lculo directo de las griegas m\'{a}s importantes}
\subsection{Delta}

Es la variacio\'{o}n del precio de la opci\'{o}n frente al subyacente (valor spot). Esto es, la derivada parcial del precio con respecto a 
$S_{t}$.
\newline

\begin{eqnarray}
	\Delta\equiv\frac{\partial C}{\partial S_{t}} & = & e^{-\delta\tau}\Phi\left(d_{+}\right) + 
								S_{t}e^{\-\delta\tau}\Phi^{'}\left(d_{+}\right)
								\frac{\partial d_{+}}{\partial S_{t}} - Ke^{-r\tau}
								\Phi^{'}\left(d_{-}\right)\frac{\partial d_{-}}{\partial S_{t}} 
								\nonumber \\
					& = & e^{-\delta\tau}\Phi\left(d_{+}\right) + 
						\frac{1}{S_{t}\sigma\sqrt{\tau}}\left[S_{t}d^{-\delta\tau}
						\Phi^{'}\left(d_{+}\right)-Ke^{-r\tau}\Phi^{'}\left(d_{-}\right)\right] 
								\nonumber 
\end{eqnarray}
\newline

Para utilizarlo en desarrollos posteriores vamos a demostrar que $S_{t}e^{-\delta\tau}\Phi^{'}\left(d_{+}\right) - 
Ke^{-r\tau}\Phi^{'}\left(d_{-}\right)$ es identicamente $0$.
\newline

\begin{eqnarray*}
	\lefteqn{\frac{1}{\sqrt{2\pi}}\left(S_{t}e^{-\delta\tau-\frac{1}{2}d_{+}^{2}}-Ke^{-r\tau-\frac{1}{2}d_{-}^{2}}\right) =} \\
	  &  & \frac{1}{\sqrt{2\pi}}\left\{ S_{t}e^{-\delta\tau-\frac{1}{2}\left[\frac{1}{\sigma\sqrt{\tau}}\ln\left(\frac{S_{t}} 
	{Ke^{-\left(r-\delta\right)\tau}}\right)+\frac{1}{2}\sigma\sqrt{\tau}\right]^{2}} - 
	 Ke^{-r\tau-\frac{1}{2}\left[\frac{1}{\sigma\sqrt{\tau}}\ln\left(\frac{S_{t}}{Ke^{-\left(r-\delta\right)\tau}}
	\right)-\frac{1}{2}\sigma\sqrt{\tau}\right]^{2}}\right\} = \\
	  &  & \frac{1}{\sqrt{2\pi}}e^{-\frac{1}{2}\left[\frac{1}{\sigma^{2}\tau}\ln^{2}\left(\frac{S_{t}}{Ke^{\left(r-\delta\right)
	\tau}}\right) + \frac{1}{4}\sigma^{2}\tau\right]}\cdot\left[S_{t}e^{-\delta\tau-\frac{1}{2}\ln\left(\frac{S_{t}}
	{Ke^{-\left(r-\delta\right)\tau}}\right)}-
	Ke^{-r\tau+\frac{1}{2}\ln\left(\frac{S_{t}}{Ke^{-\left(r-\delta\right)\tau}}\right)}\right] = \\
	  &  & \frac{1}{\sqrt{2\pi}}e^{-\frac{1}{2}\left[\frac{1}{\sigma^{2}\tau}\ln^{2}\left(\frac{S_{t}}{Ke^{\left(r-\delta\right)
	\tau}}\right) + \frac{1}{4}\sigma^{2}\tau\right]}\cdot\left[S_{t}e^{-\delta\tau}\sqrt{\frac{K}{S_{t}}}e^{-\frac{1}{2}
	\left(r-\delta\right)\tau}-Ke^{-r\tau}\sqrt{\frac{S_{t}}{K}}e^{\frac{1}{2}\left(r-\delta\right)\tau}\right] = \\
	  &  & \frac{1}{\sqrt{2\pi}}e^{-\frac{1}{2}\left[\frac{1}{\sigma^{2}\tau}\ln^{2}\left(\frac{S_{t}}{Ke^{\left(r-\delta\right)
	\tau}}\right) + \frac{1}{4}\sigma^{2}\tau\right]}\cdot\sqrt{S_{t}K}\cdot
	\underbrace{\left(e^{-\delta\tau-\frac{1}{2}r\tau+\frac{1}{2}
	\delta\tau}-e^{-r\tau+\frac{1}{2}r\tau-\frac{1}{2}\delta\tau}\right)}_{=0} = 0 \\
	Q.E.D
\end{eqnarray*}
\newline

Por tanto volviendo al c\'{a}lculo de la delta y usando este resultado intermedio:
\newline

\begin{equation}
	\Delta_{c}\equiv\frac{\partial C}{\partial S_{t}} = e^{-\delta\tau}\Phi\left(d_{+}\right)
\end{equation}
\newline

\subsection{Gamma}
El c\'{a}lculo de la Gamma es bastante inmediato, s\'{o}lo hay que volver a derivar la expresi\'{o}n obtenida de la delta, otra vez, con respecto
a $S_{ŧ}$.
\newline

\begin{eqnarray}
	\Gamma \equiv \frac{\partial \Delta_{c}}{\partial S_{t}^{2}} = \frac{\partial^{2}C}{\partial S_{t}^{2}} & = & 
	e^{-\delta\tau}\Phi^{'}\left(d_{+}\right)\frac{\partial d_{+}}{\partial S_{t}} \nonumber \\
	& = & \frac{1}{\sigma\sqrt{\tau}}\frac{e^{-\delta\tau}}{S_{t}}\Phi^{'}\left(d_{+}\right)
\end{eqnarray}
\newline

\subsection{Vega}
Es la variaci\'{o}n del precio de la opci\'{o}n con respecto a la volatilidad del subyacente.
\newline

\begin{eqnarray*}
	\lefteqn{V \equiv \frac{\partial C}{\partial \sigma} = S_{t}e^{-\delta\tau}\Phi^{'}\left(d_{+}\right)
	\frac{\partial d_{+}}{\partial \sigma} + Ke^{-r\tau}\Phi^{'}\left(d_{-}\right) = } \\
	 & & S_{t}e^{-\delta\tau}\Phi^{'}\left(d_{+}\right)\cdot\left[-\frac{1}{\sigma^{2}\sqrt{\tau}}\ln\left(\frac{S_{t}}
	{Ke^{-\left(r-\delta\right)\tau}}\right)+\frac{1}{2}\sqrt{\tau}\right] -  \\
	 & & Ke^{-r\tau}\Phi^{'}\left(d_{-}\right)\cdot\left[-\frac{1}{\sigma^{2}\sqrt{\tau}}\ln\left(\frac{S_{t}}
	{Ke^{-\left(r-\delta\right)\tau}}\right)-\frac{1}{2}\sqrt{\tau}\right] = \\
	 & & -\frac{1}{\sigma^{2}\sqrt{\tau}}\ln\left(\frac{S_{t}}{Ke^{-\left(r-\delta\right)\tau}}\right)\cdot
	\underbrace{\left(S_{t}e^{-\delta\tau}\Phi^{'}\left(d_{+}\right)-Ke^{-r\tau}\Phi\left(d_{-}\right)\right)}_{=0} + \\
	 & & \frac{1}{2}\sqrt{\tau}\left(S_{t}e^{-\delta\tau}\Phi^{'}\left(d_{+}\right) + 
	Ke^{-r\tau}\Phi^{'}\left(d_{-}\right)\right)
\end{eqnarray*}
\newline

Por tanto la vega es igual a:
\newline

\begin{equation}
	 V \equiv \frac{\partial C}{\partial\sigma} = \frac{1}{2}\sqrt{\tau}\left(S_{t}e^{-\delta\tau}\Phi^{'}\left(d_{+}\right)+
		Ke^{-r\tau}\Phi^{'}\left(d_{-}\right)\right)
\end{equation}
\newline

Vamos a desarrollora el resultado anterior para simplificar la expresi\'{o}n de la vega. Para ello vamos a utilizar
$S_{t}e^{-\delta\tau}\Phi^{'}\left(d_{+}\right)+Ke^{-r\tau}\Phi^{'}\left(d_{-}\right) = 0$ demostrado l\'{i}neas arriba, en el c\'{a}lculo
de la delta.
\newline

\begin{eqnarray}
	\lefteqn{\frac{1}{2}\sqrt{\tau}\left(S_{t}e^{-\delta\tau}\Phi^{'}\left(d_{+}\right) + 
	Ke^{-r\tau}\Phi^{'}\left(d_{-}\right)\right) = } \nonumber \\
	 & & \frac{1}{2}\sqrt{\tau}\left\{\left[S_{t} e^{-\delta\tau}\Phi^{'}\left(d_{+}\right) + 
	Ke^{-r\tau}\Phi^{'}\left(d_{-}\right)\right] + \underbrace{\left[S_{t} e^{-\delta\tau}\Phi^{'}\left(d_{+}\right) -
	Ke^{-r\tau}\Phi^{'}\left(d_{-}\right)\right]}_{=0}\right\} = \nonumber \\
	 & & S_{t}\sqrt{\tau}\Phi^{'}\left(d_{+}\right)e^{-\delta\tau} \nonumber
\end{eqnarray}
\newline

Por tanto la Vega vale.
\newline

\begin{equation}
	V \equiv \frac{\partial C}{\partial\sigma} = S_{t}\sqrt{\tau}\Phi^{'}\left(d_{+}\right)e^{-\delta\tau}
\end{equation}
 
\end{document}
